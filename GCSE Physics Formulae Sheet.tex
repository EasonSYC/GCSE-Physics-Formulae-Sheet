\documentclass[8pt]{article}
\usepackage{allan-eason}

\usepackage{siunitx}

\usetikzlibrary{positioning}
\usetikzlibrary{svg.path}

\graphicspath{ {./images/} }

\newcommand{\MeanSymb}{\textbf{Meaning of Symbols (and Units):}\ }
\newcommand{\WordExpl}{\textbf{Word Explanation:}\ }
\newcommand{\DeriForm}{\textbf{Derived Formulae:}\ }
\newcommand{\Note}{\textbf{Note:}\ }


\newcommand{\Date}{\today}
\newcommand{\Name}{GCSE Physics Formulae Sheet}
\newcommand{\Subname}{Eason's Quantitative Toolbox}
\newcommand{\Title}{\Name\\ \Large{\Subname}}

\newcommand{\Author}{Yicheng Shao}

\author{\Author}
\title{\Title}
\date{Version 1. \Date}

\lhead{\Name}

\begin{document}

	\maketitle

	\tableofcontents

    \section*{What is this and why this?}
        Most calculation error in GCSE Physics exams is caused by not knowing the formulae, not knowing the derived form and not knowing the meaning of the symbols. I made this document based on the CIE IGCSE Physics (9-1) Syllabus from 2023 onwards. I hope this could help with your IGCSE studies!

        I am also an IGCSE student so errors are inevitable in this document. Feel free to email \href{eason.syc@icloud.com}{eason.syc@icloud.com} to point out any mistakes or submit an issue on the GitHub page!

        P.S. I want to try my best to include quantitative formulae for qualitative requirements at GCSE, but they usually require more Mathematical tools so it is normal to have unexpected symbols.

    \section{Mechanics}
        \subsection{Kinematics}
            \begin{enumerate}
                \item \textbf{Definition of speed:}
                \[
                    v = \frac{\Delta s}{\Delta t}.
                \]

                \MeanSymb \(v\) stands for speed, \unit{\metre\per\second}; \(\Delta s\) stands for distance elapsed, \unit{\metre}; \(\Delta t\) stands for time elapsed, \unit{\second}.

                \WordExpl Speed equals distance covered per unit time.

                \DeriForm \(\Delta s = v \Delta t, t = \frac{\Delta s}{v}\).

                \Note Speed and distance are both scalars.

                \item \textbf{Definition of velocity:}
                \[
                    \vect{v} = \frac{\Delta \vect{s}}{\Delta t}.
                \]

                \MeanSymb \(\vect{v}\) stands for velocity, \unit{\metre\per\second}; \(\Delta \vect{s}\) stands for displacement, \unit{\metre}; \(\Delta t\) stands for time elapsed, \unit{\second}.

                \WordExpl Velocity equals the displacement over the time elapsed.

                \DeriForm \(\Delta \vect{s} = \vect{v} \Delta t, \Delta t = \frac{\Delta \vect{s}}{\vect{v}}\).

                \Note Velocity and displacement are both vectors.
            
                \item \textbf{Definition of average speed:}
                \[
                    \bar{v} = \frac{s}{t}
                \]

                \MeanSymb \(\bar{v}\) stands for average speed, \unit{\metre\per\second}; \(s\) stands for distance, \unit{\metre}; \(t\) stands for time, \unit{\second}.

                \WordExpl Average speed equals the total distance over total time.

                \DeriForm \(s = \bar{v}t, t = \frac{s}{\bar{v}}\).

                \Note Difference between average speed and speed: one is over a period, another is at a certain time.

                \item \textbf{Definition of acceleration:}
                \[
                    \vect{a} = \frac{\Delta \vect{v}}{\Delta t}.
                \]

                \MeanSymb \(\vect{a}\) stands for acceleration, \unit{\metre\per\second\squared}; \(\Delta \vect{v}\) stands for change in velocity, \unit{\metre\per\second}; \(\Delta t\) stands for time elapsed, \unit{\second}.

                \WordExpl Acceleration is the rate of change in velocity.

                \DeriForm \(\Delta \vect{v} = \Delta t \vect{a}, \Delta t = \frac{\Delta \vect{v}}{\vect{a}}\),
                \[
                    \vect{v} = \vect{v_0} + \vect{a}t
                \]

                \Note Acceleration is (usually) a vector.

            \end{enumerate}

            \textbf{You are not expected to know the \(suvat\) equation, but please draw a \(\vect{v}-t\) graph if necessary.}

        \subsection{Statics and Energy}
            \begin{enumerate}
                \item \textbf{Weight:}
                \[
                    \vect{W} = m \vect{g}.
                \]

                \MeanSymb \(\vect{W}\) stands for the weight (a force), \unit{\newton}; \(m\) stands for mass, \unit{\kilogram}; \(\vect{g}\) stands for gravitational acceleration or gravitational field strength, \unit{\metre\per\second\squared} or \unit{\newton\per\kilogram}.

                \WordExpl The gravitational acceleration is the gravitational force (weight) per unit time (and equivalent to gravitational acceleration).

                \DeriForm \(\vect{g} = \frac{\vect{W}}{m}, m = \frac{\vect{W}}{\vect{g}}\).

                \Note Weight and gravitational acceleration/gravitational field strength are both vectors.

                \item \textbf{Density:}
                \[
                    \rho = \frac{m}{V}
                \]

                \MeanSymb \(\rho\) stands for density, \unit{\kilogram \per \metre\cubed}; \(m\) stands for mass, \unit{\kilogram}; \(V\) stands for volume, \unit{\metre\cubed}.

                \WordExpl The density is mass per unit volume.

                \DeriForm \(m = \rho V, V = \frac{\rho}{m}\).

                \Note Density is a scalar, and it is a property of a material (usually), which could also determine floating and sinking.
                
                \item \textbf{Definition of pressure (General):}
                \[
                    p = \frac{F}{A}.
                \]

                \MeanSymb \(p\) stands for pressure, \unit{\pascal} or \unit{\newton\per\metre\squared}; \(F\) stands for the magnitude of the normal force, \unit{\newton}; \(A\) stands for area, \unit{\metre\squared}.

                \WordExpl Pressure is the magnitude of the force exerted per unit area.

                \DeriForm \(F = pA, A = \frac{F}{p}\).

                \Note Though force is a vector, pressure is a scalar. This will be more significant in the next formula.

                \item \textbf{Pressure (Liquid and Prism):}
                \[
                    \Delta p = \rho g \Delta h.
                \]

                \MeanSymb \(\Delta p\) stands for change in pressure, \unit{\pascal}; \(\rho\) stands for the density, \unit{\kilogram\per\metre\cubed}; \(g\) stands for gravitational acceleration, \unit{\newton \per \kilogram}; \(\Delta h\) stands for difference in depth/height, \unit{\metre}.

                \WordExpl The change in the pressure (over a depth) is the gravitational acceleration times the density times the change in depth.

                \DeriForm \(\rho = \frac{\Delta p}{g \Delta h}, \Delta h = \frac{\Delta p}{\rho g}\).

                \Note Pressure is exerted in all directions in the liquid.

                \item \textbf{Definition of work:}
                \[
                    W = \Delta E = \vect{F} \cdot \vect{d} = Fd \cos \theta.
                \]

                \MeanSymb \(W\) stands for work, \unit{\joule} or \unit{\newton \metre}; \(\Delta E\) stands for change in energy, \unit{\joule}; \(\vect{F}\) stands for force, \unit{\newton}; \(\vect{d}\) stands for displacement, \unit{\metre}; \(\theta\) stands for the angle between the force and the direction of travel, \unit{\degree} or \unit{\radian} (note that radians just a dimension of 1).

                \WordExpl Work is energy transferred. Work is the force times distance travelled in the direction of the force.

                \DeriForm \(F = \frac{W}{d \cos \theta}, d = \frac{W}{F \cos \theta}\).

                \Note Usually the \(\cos \theta\) can be ignored, but please remember the distance travelled in the direction of the force.

                \item \textbf{Kinetic energy:}
                \[
                    E_k = \frac{1}{2} m v^2.
                \]

                \MeanSymb \(E_k\) stands for kinetic energy, \unit{\joule}; \(m\) stands for mass, \unit{\kilogram}; \(v\) stands for speed or magnitude of velocity, \unit{\metre\per\second}.

                \WordExpl The kinetic energy is half the mass times energy squared.
                
                \DeriForm \(v = \sqrt{\frac{2 E_k}{m}}\).

                \Note Energy is always a scalar. It makes no difference using speed or dot product of velocity here. Note that \(\Delta E_k \neq \frac{1}{2} m \Delta v^2\).

                \item \textbf{Gravitational potential energy (in a uniform gravitational field):}
                \[
                    \Delta E_p = m g \Delta h.
                \]

                \MeanSymb \(E_p\) stands for gravitational potential energy, \unit{\joule}; \(m\) stands for mass, \unit{\kilogram}; \(g\) stands for gravitational acceleration (magnitude), \unit{\metre\per\second\squared}; \(\Delta h\) stands for change in height, \unit{\metre}.

                \WordExpl The change in gravitational potential energy is the mass times the gravitational field strength times the change in the height.

                \DeriForm \(m = \frac{\Delta E_p}{g \Delta h}, g = \frac{\Delta E_p}{m \Delta h}, \Delta h = \frac{\Delta E_p}{mg} = \frac{\Delta E_p}{W}, E_p = W \Delta h\).

                \Note This formula only holds in a uniform gravitational field, there is another one in more complex gravitational fields (e.g. Newtonian Gravity and Einstein's General Relativity).

                \item \textbf{Definition of power:}
                \[
                    P = \frac{\Delta E}{\Delta t}.
                \]

                \MeanSymb \(P\) stands for power, \unit{\watt} or \unit{\joule\per\second}; \(\Delta E\) stands for energy transferred \(= W\), \unit{\joule}; \(\Delta t\) stands for time elapsed, \unit{\second}.

                \WordExpl Power is the rate of energy transferred.

                \DeriForm \(P = \frac{W}{\Delta t}, \Delta E = W = P \Delta t, t = \frac{W}{P} = \frac{\Delta E}{P}\),
                \[
                    P = \vect{F} \cdot \vect{v}.
                \]

                \Note This formula also works for electrical power.

                \item \textbf{Efficiency:}
                \[
                    \eta = \frac{P_{\text{useful}}}{P_{\text{total}}}.
                \]

                \MeanSymb \(\eta\) stands for efficiency, dimension of \(1\) without unit; \(P_{\text{useful}}\) stands for useful power (output), \unit{\watt}; \(P_{\text{total}}\) stands for total power (input), \unit{\watt}.

                \WordExpl The efficiency is the percentage/proportion of useful energy/power output to the total energy/power input.

                \DeriForm
                \[
                    \eta = \frac{W_{\text{useful}}}{W_{\text{total}}}.
                \]

                \Note You can times \(100\%\) which is basically \(1\) to get a percentage.

            \end{enumerate}

        \subsection{Effect of Forces}
            \begin{enumerate}
                \item \textbf{Definition of momentum:}
                \[
                    \vect{p} = m\vect{v}.
                \]

                \MeanSymb \(\vect{p}\) stands for momentum, \unit{\kilogram \metre \per \second}; \(m\) stands for mass, \unit{\kilogram}; \(\vect{v}\) stands for velocity, \unit{\metre\per\second}.

                \WordExpl Momentum is the product of mass and its velocity.

                \DeriForm \(m = \frac{\vect{p}}{\vect{v}}, \vect{v} = \frac{\vect{p}}{m}\).

                \Note Momentum itself is a vector and has a direction.
                
                \item \textbf{Newton's 2nd Law:}
                \[
                    \vect{F} = \frac{\Delta \vect{p}}{\Delta t}.
                \]

                \MeanSymb \(\vect{F}\) stands for force, \unit{\newton}; \(\Delta \vect{p}\) stands for change in momentum, \unit{\kilogram \metre \per \second}; \(\Delta t\) stands for time elapsed, \unit{\second}.

                \WordExpl Force is equal to the rate of change in momentum.

                \DeriForm \(\Delta \vect{p} = \vect{F} \Delta t, \Delta t = \frac{\Delta \vect{p}}{\vect{F}}\),
                \[
                    \vect{F} = m \vect{a}.
                \]

                \Note This is a very important equation in physics, and it could lead to discussions about inertial mass/gravitational mass, Lagrange -- d'Almbert's Principle (turning non-inertial frames into inertial ones), special relativity, etc.

                \item \textbf{Impulse:}
                \[
                    \vect{I} = \Delta \vect{p}.
                \]

                \MeanSymb \(\vect{I}\) stands for impulse, \unit{\kilogram \metre \per \second}; \(\Delta \vect{p}\) stands for change in momentum, \unit{\kilogram \metre \per \second}.

                \WordExpl Impulse is equal to the change in momentum.

                \DeriForm The following is derived from Newton's 2nd Law:
                \[
                    \vect{I} = \vect{F} \Delta t.
                \]

                \Note This is only meaningful if momentum is conserved - just like work and energy.

                \item \textbf{Moment:}
                \[
                    \vect{M} = \vect{r} \times \vect{F}.
                \]

                \MeanSymb \(\vect{M}\) stands for the moment (of a force), \unit{\newton \metre} (according to SI standards, we don't write it as a \unit{\joule}); \(\vect{r}\) stands for the position vector of the force, \unit{\metre}; \(\vect{F}\) stands for the force, \unit{\newton}.

                \WordExpl The magnitude of the moment of a force is equal to the magnitude of the force times the perpendicular distance between the pivot and the line of action of the force.

                \DeriForm To be simple, denote \(d\) as the perpendicular distance between the pivot and the line of action of the force, \unit{\metre}, then we have \(M = Fd, F = \frac{M}{d}, d = \frac{M}{F}\).

                \Note I wrote this in terms of vector and their cross-product just for the sake of science but this is not required at all.

                \item \textbf{Hooke's Law:}
                \[
                    \vect{F} = k \vect{x}.
                \]

                \MeanSymb \(\vect{F}\) stands for force, \unit{\newton}; \(k\) stands for the spring constant, \unit{\newton \per \metre}; \(\vect{x}\) stands for the extension (vector).

                \WordExpl The force to extend or compress a spring (within the limit of linearity) is perpendicular to the extension.

                \DeriForm \(k = \frac{\vect{F}}{\vect{x}}, \vect{x} = \frac{\vect{F}}{k}\).

                \Note Remember to use extension for the \(\vect{x}\) not the total length.

                \item \textbf{Circular Motion:} (qualitative explanation required)
                \[
                    F = \frac{mv^2}{r}.
                \]

                \MeanSymb \(F\) stands for the centripetal force (i.e.) the force pointing to the centre of the circular motion (perpendicular to velocity), \unit{\newton}; \(m\) stands for mass, \unit{\kilogram}; \(v\) stands for speed, \unit{\metre\per\second}; \(r\) stands for radius, \unit{\metre}.

                \WordExpl The centripetal force for a circular motion is proportional to its mass, the square of its speed, and inversely proportional to its radius.

                \DeriForm \(r = \frac{mv^2}{F} m = \frac{Fr}{v^2}, v = \sqrt{\frac{Fr}{m}}\).

                \Note The quantitative form is not required but you are required to know the qualitative aspect. Circular motion is accelerating!
            \end{enumerate}

    \section{Thermal Physics}
        \subsection{Ideal Gas}
            \begin{enumerate}
                \item \textbf{Boyle's Law:}
                \[
                    pV = \mathrm{const.}
                \]

                \MeanSymb \(p\) stands for pressure, \unit{\pascal}; \(V\) stands for volume, \unit{\metre \cubed}.

                \WordExpl The pressure of a gas is inversely proportional to its volume given that its temperature remains the same.

                \DeriForm \(p_1 V_1 = p_2 V_2\).

                \Note This only remains true if the temperature is constant.
                
                \item \textbf{Charles's Law:} (it popped out on the MTR, and stating this will get you a mark!)
                \[
                    \frac{V}{T} = \mathrm{const.}
                \]

                \MeanSymb \(V\) stands for volume, \unit{\metre \cubed}; \(T\) stands for temperature, \unit{\kelvin}.

                \WordExpl The volume of a gas is directly proportional to its temperature given that its pressure remains the same.

                \DeriForm \(\frac{V_1}{T_1} = \frac{V_2}{T_2}, \frac{V_1}{V_2} = \frac{T_1}{T_2}\).

                \Note This only remains true if pressure is a constant.
                
                \item \textbf{Gay-Lussac's Law:} (just for the sake of knowing it)
                \[
                    \frac{p}{T} = \mathrm{const.}
                \]

                \MeanSymb \(p\) stands for pressure, \unit{\pascal}; \(T\) stands for temperature, \unit{\kelvin}.

                \WordExpl The pressure of a gas is directly proportional to its temperature given that its volume remains the same.

                \DeriForm \(\frac{p_1}{T_1} = \frac{p_2}{T_2}, \frac{p_1}{p_2} = \frac{T_1}{T_2}\).

                \Note This only remains true if the volume is a constant.

                \item \textbf{Avogardo's Equation (Chemistry!):}
                
                No that's not my work.

                \item \textbf{Ideal Gas Law:} (a.k.a. Clapeyron Equation, the boss, just for reference)
                \[
                    pV = nRT = Nk_{B}T
                \]

                \MeanSymb \(p\) stands for pressure, \unit{\pascal}; \(V\) stands for volume, \unit{\metre\cubed}; \(n\) stands for moles, \unit{\mole}; \(R\) stands for the ideal gas constant (which is equal to the Boltzmann constant times the Avogardo constant), \(R = k_B N_A = \qty{8.31}{\joule \per \kelvin \per \mole}\); \(T\) stands for the temperature, \unit{\kelvin}; \(N\) stands for the number of gas molecules, no unit; \(k_B\) stands for the Boltzmann constant, \(k_B = \qty{1.38 e -34}{\joule \per \kelvin}\).

                \Note This is the boss but it is A-Level knowledge, so just for the sake of knowing it.
                
                \item \textbf{van der Waals Equation (a more boss equation)}
                
                No, I won't talk about this, this is too much for now.
            \end{enumerate}

        \subsection{Temperature}
            \begin{enumerate}
                \item \textbf{Conversion between kelvin and degree celsius:}
                \[
                    T \unit{\per \kelvin} = \theta \unit{\per \degreeCelsius} + 273(.15).
                \]

                \MeanSymb \(T\) stands for (thermodynamic) temperature in kelvin, \unit{\kelvin}; \(\theta\) stands for temperature in degrees Celsius, \unit{\degreeCelsius}.

                \WordExpl The temperature in kelvin is equal to the temperature in degrees Celsius plus \(273.15\).

                \DeriForm \(\theta \unit{\per \degreeCelsius} = T \unit{\per \kelvin} - 273(.15)\).

                \Note Please note that in all the ideal gas equations you need to use the thermodynamic temperature, but in the following equation you do not need to convert, as the change in one degree Celsius equals the change in one kelvin.

                \item \textbf{Thermal capacity:}
                \[
                    Q = m c \Delta T.
                \]

                \MeanSymb \(Q\) stands for thermal energy transferred, \unit{\joule}; \(m\) stands for mass, \unit{\kilogram}; \(c\) stands for thermal capacity, \unit{\joule \per \kilogram \per \kelvin} or \unit{\joule \per \kilogram \per \degreeCelsius}; \(\Delta T\) stands for change in temperature, \unit{\kelvin} or \unit{\degreeCelsius}.

                \WordExpl The thermal capacity is defined as the heat energy transferred per unit mass per unit change in temperature.

                \DeriForm \(m = \frac{Q}{c \Delta T}, c = \frac{Q}{m \Delta T}, \Delta T = \frac{Q}{mc}\).

                \Note Thermal capacity is a property of a material. It doesn't matter whether you calculate with degrees Celsius or kelvin, but make sure you use the same unit to calculate the temperature change.

                \item \textbf{Latent heat:} (old syllabus, but it popped up before!)
                \[
                    Q = ml.
                \]

                \MeanSymb \(Q\) stands for thermal energy transferred, \unit{\joule}; \(m\) stands for mass, \unit{\kilogram}; \(l\) stands for specific latent heat, \unit{\joule \per \kilogram}.

                \WordExpl The latent heat is defined as the heat energy transferred per unit mass to convert a substance from one state to another state.

                \DeriForm \(m = \frac{Q}{l}, l = \frac{Q}{m}\).

                \Note Latent heat is a property of a material and the state conversion it is in.
                
            \end{enumerate}

    \section{Waves}
        \subsection{Waves}
            \begin{enumerate}
                \item \textbf{Frequency and Period:}
                \[
                    f = \frac{1}{T}.
                \]

                \MeanSymb \(f\) stands for the frequency, \unit{\hertz} or \unit{\per \second}; \(T\) stands for the period, \unit{\second}.

                \WordExpl Frequency is the reciprocal of the period.

                \DeriForm \(fT = 1, T = \frac{1}{f}\).

                \Note This is true for all waves.

                \item \textbf{The wave equation:}
                \[
                    v = f \lambda.
                \]

                \MeanSymb \(v\) stands for the wave speed, \unit{\metre \per \second}; \(f\) stands for the frequency, \unit{\hertz}; \(\lambda\) stands for the wavelength, \unit{\metre}.

                \WordExpl The wave speed of a wave is equal to its frequency times its wavelength.

                \DeriForm \(f = \frac{v}{\lambda}, \lambda = \frac{v}{f}\).

                \Note The frequency of a wave will not change (Doppler Effect! but that's an observation), so usually we can just say \(\lambda \propto v\).
            \end{enumerate}

        \subsection{Optics}
            \begin{enumerate}
                \item \textbf{Refractive index:}
                \[
                    n = \frac{c}{v}
                \]

                \MeanSymb \(n\) stands for the refractive index, dimension of \(1\); \(c\) stands for the speed of light in vacuum, \(c = \qty{3.00 e 8}{\metre \per \second}\); \(v\) stands for the speed of light in that certain medium.

                \WordExpl The refractive index of a medium is the ratio of the speed of light in a vacuum and the speed of light in that certain medium.

                \DeriForm \(c = nv, v = \frac{c}{n}\).

                \Note As the speed of light in a vacuum is the fastest thing in the world (special relativity!), \(n\) is always no smaller than one i.e. \(n \geq 1\) and \(n = 1\) iff. the medium is a vacuum. For simplicity we take \(n_{\text{air}} = 1\).
                
                \item \textbf{Snell's law:}
                \[
                    n_1 \sin \theta_1 = n_2 \sin \theta_2.
                \]

                \MeanSymb \(n_1\) stands for the refractive index in the first medium, dimension of \(1\); \(\theta_1\) stands for the angle between the incident ray and the normal (i.e. incidence angle), \(\degree\); \(n_2\) stands for the refractive index in the second medium, dimension of \(1\); \(\theta_2\) stands for the angle between the refracted ray and the normal (i.e. refraction angle), \(\degree\).

                \WordExpl The product of the refractive index and the angle between the ray and the normal is the same while the light is refracting.

                \DeriForm \(\theta_2 = \arcsin \frac{n_1 \sin \theta 1}{n_2}, n_2 = \frac{n_1 \sin \theta_1}{\sin\theta_2}\).

                \Note This can be derived from Fermat's Principle, or maybe, least action principle!
                
                \item \textbf{Critical angle:}
                \[
                    \sin c = \frac{n_{\text{quick}}}{n_{\text{slow}}}.
                \]

                \MeanSymb \(c\) stands for the critical angle, \(\degree\); \(n_{\text{quick}}\) stands for the refractive index in the quick medium, dimension of \(1\); \(n_{\text{slow}}\) stands for the refractive index in the slow medium, dimension of \(1\).

                \WordExpl The sine of the critical angle is equal to the ratio between the refractive index of the fast medium and the slow medium.

                \DeriForm \(n_{\text{slow}} = \frac{n_{\text{quick}}}{\sin c}\).

                \Note This is a special case of the general Snell's law where a \(\theta\) equals \(90 \degree\). We see \(\sin c = \frac{1}{n}\) as we take the quick \(n_{\text{quick}}\) as \(1\) (in the case of air).
            \end{enumerate}

    \section{Electricity}
        \subsection{Electrical Quantities}
            \begin{enumerate}
                \item \textbf{Definition of electromotive force (e.m.f.):}
                \[
                    E = \frac{W}{Q}.
                \]

                \MeanSymb \(E\) stands for the e.m.f., \unit{\volt} or \unit{\joule\per\coulomb}; \(W\) stands for work, \unit{\joule}; \(Q\) stands for charge, \unit{\coulomb}.

                \WordExpl The e.m.f. is equal to the electrical work done per unit charge to move it around a circuit.

                \DeriForm \(W = EQ, Q = \frac{W}{E}\).

                \Note The e.m.f. is not electromagnetic force, it is electromotive force! But not a force.

                \item \textbf{Definition of potential difference (p.d.):}
                \[
                    V = \frac{W}{Q}.
                \]

                \MeanSymb \(V\) stands for the p.d., \unit{\volt} or \unit{\joule\per\coulomb}; \(W\) stands for work, \unit{\joule}

                \WordExpl The p.d. is the work done per unit charge passing through an electric component.

                \DeriForm \(W = VQ, Q = \frac{W}{V}\).

                \Note The p.d. is effectively the same with e.m.f. -- but e.m.f. is converting other forms of energy into electrical energy while p.d. is the other way round.

                \item \textbf{Definition of current:}
                \[
                    I = \frac{Q}{t}.
                \]

                \MeanSymb \(I\) stands for the current, \unit{\ampere} or \unit{\coulomb\per\second}; \(Q\) stands for the charge, \unit{\coulomb}; \(t\) stands for time, \unit{\second}.

                \WordExpl The current is the charge flowing through a point per unit time. (Technically it should be \(\Delta Q\) and \(\Delta t\) but we don't mind here.)

                \DeriForm \(Q = It, t = \frac{Q}{I}\).

                \Note We don't care about \(I = \frac{\Delta Q}{\Delta t}\) because we only study constant current at this stage. Current is \textbf{not} a vector, but current density is (we don't need to know that yet but just FYI.)

                \item \textbf{Formula for electrical work and electrical power:}
                \[
                    P = VI, W = Pt = VIt = VQ.
                \]

                \MeanSymb \(P\) stands for power, \unit{\watt}; \(V\) stands for voltage, \unit{\volt}; \(I\) stands for current, \unit{\ampere}; \(W\) stands for work, \unit{\joule}; \(t\) stands for time, \unit{\second}.

                \WordExpl The electrical power is equal to the voltage times the current; the electrical work done is equal to the voltage times the current times the time elapsed, which is also equal to the voltage times the charge.

                \DeriForm \(V = \frac{P}{I}, I = \frac{P}{V}\).

                \Note These are derived formulae from the definition of work and the definition of the current so it shouldn't be surprising if these are true. We will see an updated version of these in the next section.
            \end{enumerate}

        \subsection{Circuits}
            \begin{enumerate}
                \item \textbf{Ohm's Law and Definition of Resistance:} (simplified version)
                \[
                    R = \frac{V}{I}.
                \]

                \MeanSymb \(R\) stands for resistance, \unit{\ohm} or \unit{\volt\per\ampere}; \(V\) stands for voltage (usually p.d.), \unit{\volt}; \(I\) stands for current, \unit{\ampere}.

                \WordExpl The resistance of an electrical component is the potential difference across it over the current passing through it.

                \DeriForm \(V = RI, I = \frac{V}{R}\).

                \Note This is a very important thing in electrical circuit calculation -- please remember this as it links some of the most important quantities together. Note that \(R \neq \frac{\Delta V}{\Delta I}\) which means resistance is not a gradient at a point!

                \item \textbf{More Power:}
                \[
                    P = VI = I^2 R = \frac{V^2}{R}.
                \]

                \MeanSymb \(P\) stands for electrical power, \unit{\watt}; \(V\) stands for voltage, \unit{\volt}; \(I\) stands for current, \unit{\ampere}; \(R\) stands for resistance, \unit{\ohm}.

                \WordExpl The power is the product of the square of current and the resistance, which is equal to the square of voltage over the resistance.

                \DeriForm Too many.

                \Note This is a direct corollary of Ohm's Law.

                \item \textbf{Kirchoff's First Law:}
                \[
                    \sum I_{\text{in}} = \sum I_{\text{out}}.
                \]

                \MeanSymb \(\sum\) stands for the sum; \(I_{\text{in}}\) stands for the current flowing into a junction, \unit{\ampere}; \(I_{\text{out}}\) stands for the current flowing out a junction, \unit{\ampere}.

                \WordExpl The current flowing into a junction is equal to the current flowing out of a junction.

                \DeriForm The parallel circuit law (which will appear later).

                \Note This is a direct corollary from the conservation of charge.

                \item \textbf{Kirchoff's Second Law:} (may be useful)
                \[
                    \sum V = \sum E.
                \]

                \MeanSymb \(\sum\) stands for the sum; \(V\) stands for the p.d., \unit{\volt}; \(E\) stands for the e.m.f., \unit{\volt}.

                \WordExpl The sum of the p.d. is equal to the sum of the e.m.f. in a closed loop.

                \DeriForm \(\sum V = 0\) if no e.m.f., \(\sum E = 0\) if no p.d.

                \Note This is a direct corollary from the conservation of energy. Notice that the loop itself is 'directional' -- that is, some 'batteries' may act as a p.d. rather than an e.m.f. on the loop.

                \item \textbf{Parallel Circuit Laws:}
                \begin{align*}
                    I &= I_1 + I_2 + \cdots + I_n,\\
                    V &= V_1 = V_2 = \cdots + V_n,\\
                    \frac{1}{R} &= \frac{1}{R_1} + \frac{1}{R_2} + \cdots + \frac{1}{R_n}.
                \end{align*}

                \MeanSymb \(I\) stands for the total current, \unit{\ampere}; \(I_i\) stands for the current passing through the \(i\)th component, \unit{\ampere}; \(V\) stands for the e.m.f., \unit{\volt}; \(V_i\) stands for the p.d. over the \(i\)th component, \unit{\volt}; \(R\) stands for the total resultant resistance, \unit{\ohm}; \(R_i\) stands for the resistance of the \(i\)th component, \unit{\ohm}.

                \WordExpl The total current in a parallel circuit is the sum of all the current in each branch; the e.m.f. of the parallel circuit is equal to the p.d. over each component; the resistance of the circuit is equal to the reciprocal of the sum of reciprocal of the resistances.

                \DeriForm Too many.

                \Note The p.d. and the e.m.f. are the same, as you are effectively measuring the same thing. The result of the current follows from the Kirchoff's Current Law and the resistance formula follows.

                \item \textbf{Series Circuit Laws:}
                \begin{align*}
                    I &= I_1 = I_2 = \cdots = I_n,\\
                    V &= V_1 + V_2 + \cdots + V_n,\\
                    R &= R_1 + R_2 + \cdots + R_n.
                \end{align*}

                \MeanSymb \(I\) stands for the total current, \unit{\ampere}; \(I_i\) stands for the current passing through the \(i\)th component, \unit{\ampere}; \(V\) stands for the e.m.f., \unit{\volt}; \(V_i\) stands for the p.d. over the \(i\)th component, \unit{\volt}; \(R\) stands for the total resultant resistance, \unit{\ohm}; \(R_i\) stands for the resistance of the \(i\)th component, \unit{\ohm}.

                \WordExpl The current in a series circuit is all the same; the e.m.f. in a series circuit is the sum of the p.d.s of the components; the resistance of a series circuit is the sum of each component's resistance.

                \DeriForm Potential divider:
                \[
                    V_{\text{out}} = V_{\text{in}} \cdot \frac{R_{\text{output}}}{R_{\text{total}}}.
                \]

                \Note The current law follows from Kirchoff's Current Law and the voltage follows from the Voltage Law, the resistance result is a corollary of them.

                \item \textbf{Resistance v.s. Length and Area}:
                \[
                    R = \frac{\rho l}{A}.
                \]

                \MeanSymb \(R\) stands for the resistance, \unit{\ohm}; \(\rho\) stands for the electrical resistivity, \unit{\ohm \metre}; \(l\) stands for the length, \unit{\metre}; \(A\) stands for the cross-sectional area, \unit{\metre\squared}.

                \WordExpl The resistance of a resistor is proportional to its length and inversely proportional to the area.

                \DeriForm \(\rho = \frac{RA}{l}, A = \frac{\rho l}{R}, l = \frac{RA}{A}\).

                \Note This formula and its calculations are not required. However, you need to know everything apart from the \(\rho\)!
            \end{enumerate}

        \subsection{Electromagnetism}
            \begin{enumerate}
                \item \textbf{Transformer:}
                \[
                    \frac{V_1}{V_2} = \frac{N_1}{N_2}.
                \]
                \MeanSymb \(V_1\) stands for the primary coil voltage, \unit{\volt}; \(V_2\) stands for the secondary coil voltage, \unit{\volt}; \(N_1\) stands for the number of coils on the primary coil, dimensionless; \(N_2\) stands for the number of coils on the secondary coil, dimensionless.

                \WordExpl The coil voltage is proportional to the number of coils on that respective coil.

                \DeriForm By conservation of energy,
                \[
                    I_1 N_1 = I_2 N_2.
                \]

                \Note This is a very simple equation, so don't get it wrong! Be careful with the corresponding primary and secondary.

                \item \textbf{Coulomb's Law:} (qualitative required)
                \[
                    \vect{F} = \frac{1}{4 \pi \varepsilon_0} \frac{Q q}{r^2} \vect{l}.
                \]
                \MeanSymb \(\vect{F}\) stands for the attraction received by \(q\) caused by the electrical field created by \(Q\), \unit{\newton}; \(\varepsilon_0\) stands for the electrical permittivity in vacuum, \(\varepsilon_0 = \qty{8.85 e -12}{\farad \per \metre}\); \(Q\) stands for the charge causing the force, \unit{\coulomb}; \(q\) stands for the charge receiving the force, \unit{\coulomb}; \(r\) stands for the distance between \(Q\) and \(q\), \unit{\metre}; \(\vect{l}\) stands for the unit vector pointing from \(Q\) to \(q\) (which basically states the direction), dimensionless.

                \WordExpl The electrostatic force between two charges is proportional to the product of those two charges and inversely proportional to the distance between them squared, with attraction if different symbols charge and repulsion if the same symbol of charge.

                \Note This is also an inverse square law! Please remember electromagnetic forces are not caused directly by the particles but by field interactions.

                \item \textbf{Electrical Field:} (qualitative (and diagram) description required, formula not required at this stage)
                \[
                    \vect{E} = \frac{\vect{F}}{q} = \frac{1}{4 \pi \varepsilon_0} \frac{Q}{r^2} \vect{l}.
                \]
                \MeanSymb \(\vect{E}\) stands for the electrical field (a vector field), \unit{\volt\per\metre}; \(\vect{F}\) stands for the attraction received by \(q\) caused by the electrical field created by \(Q\), \unit{\newton}; \(\varepsilon_0\) stands for the electrical permittivity in a vacuum, \(\varepsilon_0 = \qty{8.85 e -12}{\farad \per \metre}\); \(Q\) stands for the charge causing the force, \unit{\coulomb}; \(q\) stands for the charge receiving the force, \unit{\coulomb}; \(r\) stands for the distance between \(Q\) and \(q\), \unit{\metre}; \(\vect{l}\) stands for the unit vector pointing from \(Q\) to \(q\) (which states the direction), dimensionless.

                \WordExpl The electrical field is defined by the force received per unit charge.

                \Note \(\vect{E}\) is a vector field, which means there is a vector at any point in the space. Field lines can represent most vector fields.

                \item \textbf{Potential Difference and Electrical Field:} (not required, just me being a nerd)
                \[
                    \vect{E} = - \nabla V.
                \]

                \MeanSymb \(\vect{E}\) stands for the electrical field, \unit{\volt \per \metre}; \(\nabla\) stands for the 3-D gradient operator, \unit{\per \metre}; \(V\) stands for the voltage, \unit{\volt}.

                \WordExpl The electrical field is equal to the opposite of the gradient in voltage.

                \Note The definition of the \(\nabla\), in fact a 3-D gradient:
                \[
                    \frac{\hat{\vect{x}} \partial}{\partial x} + \frac{\hat{\vect{y}} \partial}{\partial y} + \frac{\hat{\vect{z}} \partial}{\partial z}.
                \]
                \item \textbf{Magnetic Force on Current-Carrying Conductor:} (qualitative required)
                \[
                    \vect{F} = \vect{B} I \times \vect{l}.
                \]
                \MeanSymb \(\vect{F}\) stands for the magnetic force received by the wire, \unit{\newton}; \(\vect{B}\) stands for the magnetic field (to be precise, magnetic flux density field), \unit{\tesla}; \(I\) stands for the current, \unit{\ampere}; \(\vect{l}\) stands for the vector pointing from the start of the wire to the end of wire (direction defined by the current), \unit{\metre}.

                \WordExpl The magnitude of the force the current-carrying wire receives is proportional to the strength of the magnetic field, the current and its length, and its direction will satisfy Flemming's LH rule (cross product).

                \DeriForm Let \(\theta\) be the angle between the wire and the magnetic field, we have
                \[
                    F = BIl \sin \theta.
                \]

                \Note This is cool, and this could be used to define the \(\vect{B}\) field. The cross product simply represents the direction without the need to use the Cartesian coordinate formed by Flemming's LH rule.

                \item \textbf{Lorentz Force (for a charge moving in a magnetic field):} (qualitative required)
                \[
                    \vect{F} = q\vect{v} \times \vect{B}.
                \]
                \MeanSymb \(\vect{F}\) stands for the magnetic force received by the charge, \unit{\newton}; \(q\) stands for the charge, \unit{\coulomb}; \(\vect{v}\) stands for the velocity of the charge, \unit{\metre\per\second}; \(\vect{B}\) is the magnetic field, \unit{\tesla}.

                \WordExpl The magnitude of the magnetic force received by a charge is proportional to its charge, its velocity and the strength of the magnetic field, with its direction fitting Flemming's LH rule.

                \DeriForm Let \(\theta\) be the angle between the velocity and the magnetic field, we have
                \[
                    F = qvB \sin \theta.
                \]

                \Note Based on the cross product, unsurprisingly, the Lorentz Force does not do work on the charge (as the force is always perpendicular to the velocity of charge hence no power and no work), and a charge will do circular motion in a uniform magnetic field (cool!). Note that when we use Flemming's LH rule on a charge which is negative, we need to reverse its velocity to get its current.

                \item \textbf{Magnetic Flux:} (not required at this stage)
                \[
                    \Phi = \vect{B} \cdot \vect{S}.
                \]
                \MeanSymb \(\Phi\) stands for the magnetic flux, \unit{\weber} or \unit{\volt \second} or \unit{\tesla \metre \squared}; \(\vect{B}\) stands for the magnetic flux density, \unit{\tesla}; \(\vect{S}\) stands for the surface vector (direction perpendicular to the surface), \unit{\metre\squared}.

                \WordExpl The magnetic flux is the dot product of the magnetic flux density and the surface (as the name suggests).

                \DeriForm Let \(\theta\) be the angle between the surface vector and the magnetic field, we have
                \[
                    \Phi = BS \cos \theta.
                \]

                Furthermore,
                \[
                    \Phi = \iint_{\Sigma} \vect{B} \cdot \diff \vect{S}.
                \]

                \Note This will be useful for the next part: Faraday's Law of Induction.

                \item \textbf{Faraday's Law of Induction:} (with Lenz's Law) (qualitative required)
                \[
                    E = -N \frac{\Delta \Phi}{\Delta t}
                \]
                \MeanSymb \(E\) stands for the e.m.f. created, \unit{\volt}; \(N\) stands for the number of coils, dimensionless; \(\Delta \Phi\) stands for the change in magnetic flux, \unit{\weber}; \(\Delta t\) stands for time elapsed, \unit{\second}.

                \WordExpl The electromotive force created by a coil cutting through the magnetic field lines is proportional to the number of turning coils it has and the rate of change in magnetic flux (i.e. the rate cutting across magnetic field lines), and by Lenz's Law will have a direction opposing the change causing it.

                \Note This is the reason why the generator creates a.c. current.

                \item \textbf{Poynting Vector:} (not required, stands for direction of energy transfer in an electromagnetic wave)
                \[
                    \vect{S} = \frac{1}{\mu_0} \vect{E} \times \vect{B}.
                \]

                \MeanSymb \(\vect{S}\) stands for the Poynting Vector, \unit{\watt \per \metre \squared}; \(\mu_0\) stands for the vacuum magnetic permeability, \(\mu_0 = \qty{1.26 e -6}{\newton \per \ampere \squared}\); \(\vect{E}\) stands for the electrical field, \unit{\volt \per \metre}l; \(\vect{B}\) stands for the magnetic field, \unit{\tesla}.

                \WordExpl The Poynting Vector is defined as the cross product of the electrical field and the magnetic field over the vacuum magnetic permeability.

                \Note It represents the directional energy flux (the energy transfer per unit area per unit time), and demonstrates why an electromagnetic wave is a transverse wave. 
                
                \item \textbf{Maxwell's Equations:} (the boss of everything, not required)
                
                \begin{align*}
                    \nabla \cdot \vect{E} &= \frac{\rho}{\varepsilon_0},\\
                    \nabla \cdot \vect{B} &= 0,\\
                    \nabla \times \vect{E} &= - \frac{\partial \vect{B}}{\partial t},\\
                    \nabla \times \vect{B} &= \mu_0 \left(\vect{J} + \varepsilon_0 \frac{\partial \vect{E}}{\partial t}\right).
                \end{align*}

                \MeanSymb \(\nabla \cdot\) stands for the divergence, which means 'the source of a field', \unit{\per\metre}; \(\nabla \times\) stands for the curl, which means 'the circulation of a field', \unit{\per\metre}; \(\varepsilon_0\) stands for vacuum electrical permittivity, \unit{\farad\per\metre}; \(\mu_0\) stands for the vacuum magnetic permeability, \unit{\newton \per \ampere \squared}; \(\rho\) stands for the charge density, \unit{\coulomb\per\metre\cubed}; \(t\) stands for time, \unit{\second}; \(\vect{E}\) stands for the electrical field, \unit{\volt\per\metre}; \(\vect{B}\) stands for the magnetic flux density, \unit{\tesla}; \(\vect{J}\) stands for the current density, \unit{\ampere\per\metre\squared}.

                \WordExpl Electrical field has a source, but a magnetic field does not; a magnetic field causes an effect on electrical fields, and vice versa.
                \begin{enumerate}
                    \item \textbf{Gauss's Law:} Electrical field has a source which is caused by a charge.
                    \item \textbf{Gauss's Law for Magnetism:} Magnetic field does not have a source (i.e. magnetic monopole doesn't exist).
                    \item \textbf{Faraday's Law of Induction:} Electrical field is caused by a changing magnetic field, and its strength is proportional to the rate of change in magnetic flux.
                    \item \textbf{Ampere's Circuital Law:} Magnetic field is caused by a changing electrical field (but remember that electrical charge can exist individually and that's why the current density term exists).
                \end{enumerate}

                \DeriForm The integration version: (longer)
                \begin{align*}
                    \oiint_{\partial \Omega} \vect{E} \cdot \diff \vect{S} &= \frac{1}{\varepsilon_0} \iiint_{\Omega} \rho \diff V,\\
                    \oiint_{\partial \Omega} \vect{B} \cdot \diff \vect{S} &= 0,\\
                    \oint_{\partial \Sigma} \vect{E} \cdot \diff \vect{l} &= - \frac{\diff}{\diff t} \iint_{\Sigma} \vect{B} \cdot \diff \vect{S},\\
                    \oint_{\partial \Sigma} \vect{B} \cdot \diff \vect{l} &= \mu_0 \left(\iint_{\Sigma} \vect{J} \cdot \diff \vect{S} + \varepsilon_0 \frac{\diff}{\diff t} \iint_{\Sigma} \vect{E} \cdot \diff \vect{S}\right).
                \end{align*}

                \Note Well I can't calculate all this but it's cool, isn't it? Just like Einstein's field equation. There is no reason (so far) for magnetic monopole not to exist but we haven't found it yet. The equations can be very slightly adjusted for a magnetic field to exist. In terms of symmetry, I would love magnetic monopoles to exist!
            \end{enumerate}

            \textbf{Electromagnetism is one of the coolest parts of Physics, but due to the limitation of Mathematical tools, we can only stop here. (Sorry if this confused you, you just need to know the first one (transformer) for GCSE!) Well actually I just wrote the Physics I like.}

    \section{Space Physics}
        \subsection{Orbits}
            \textbf{The orbit period:}
            \[
                T = \frac{2 \pi r}{v}.
            \]

            \MeanSymb \(T\) stands for the orbit period, \unit{\second}; \(r\) stands for the orbit radius, \unit{\metre}; \(v\) stands for the orbit speed, \unit{\metre\per\second}.

            \WordExpl The orbit period is the circumference of the orbit over the orbit speed (trivial).

            \DeriForm \(v = \frac{2\pi r}{T}, r = \frac{Tv}{2\pi}\).

            \Note This can be used to derive the escape velocity (together with Newton's Law of Gravity).

        \subsection{Hubble's Constant}
            \begin{enumerate}
                \item \textbf{The Hubble's Constant:}
                \[
                    H_0 = \frac{v}{d}.
                \]

                \MeanSymb \(H_0\) stands for the Hubble Constant, \(H_0 \approx \qty{2.2 e -18}{\per \second}\); \(v\) stands for the receding velocity, \unit{\metre\per\second}; \(d\) stands for the distance, \unit{\metre}.

                \WordExpl The Hubble Constant is defined as the ratio of the speed at which the galaxy is moving away from the Earth to its distance from the Earth.

                \DeriForm \(v = H_0 d, d = \frac{v}{H_0}\).

                \Note According to the syllabus, you need to know the approximate value of the Hubble Constant.

                \item \textbf{Edge of Time:}
                \[
                    t = \frac{1}{H_0}.
                \]

                \MeanSymb \(t\) stands for the age of the universe, \unit{\second}, \(H_0\) stands for the Hubble Constant, \unit{\per\second}.

                \WordExpl The Hubble Constant is equal to the reciprocal of the age of the universe.

                \DeriForm Not really, but \(t H_0 = 1\).

                \Note \(t \approx \qty{1.38 e 10}{\mathrm{yr}}\).
            \end{enumerate}

    \section*{Afterwords}
        This sheet took me some time to populate, but I would genuinely like to share my understanding of GCSE Physics (and beyond) with all of you.
        
        My deepest thanks to all the Physics teachers I have met and my friends who helped and supported me with producing this.
        
        I would like to give special thanks to Mr Blundell, my current Physics teacher, and Dr Yun, my tutor for providing me with all this knowledge.

        Finally, I hope this helped you to gain a better understanding of Physics. Feel free to email \href{eason.syc@icloud.com}{eason.syc@icloud.com} to send me feedback.

\end{document}